
%\documentclass[10pt,aspectratio=149,handout]{beamer}
\documentclass[10pt,aspectratio=149]{beamer}

\usepackage[T1]{fontenc}
\usepackage[semibold]{sourcesanspro} % good font
\usepackage[scaled=0.9]{beramono}

\usepackage{pgfpages}
\usepackage{bm}
\usepackage{fancyvrb}
\usepackage{tikz}
%\usepackage{pgfplots}
\usepackage{xspace}

\usepackage{alltt}
\usepackage{latexcolors}
%\usepackage{version}
%\usepackage{etoolbox}
%\usepackage{tcolorbox}
%\usepackage{boxproof}
%\usepackage{mathpartir}

\usepackage[framemethod=tikz,roundcorner=4pt]{mdframed}
%\usepackage[absolute,overlay]{textpos}

\usetikzlibrary{arrows,backgrounds,shadows,shapes}
\tikzstyle{every picture}+=[remember picture]

%---------------------------------------------------------
%% Beamer customization

%\setbeamercolor{math text}{fg=tuftsblue}
%\setbeamercolor{math text}{fg=copperrose}
\setbeamercolor{math text}{fg=gray}
\setbeamercolor{alerted text}{fg=mordantred19}
\setbeamercolor{frametitle}{fg=black}
\setbeamercolor{title}{fg=black}
\setbeamercolor{enumerate item}{fg=black}
\setbeamercolor{itemize item}{fg=black}
\setbeamercolor{itemize subitem}{fg=black}
\setbeamercolor{subtitle}{fg=mordantred19}
\setbeamercolor{theorem text}{fg=mordantred19} 


\setbeamerfont{frametitle}{series=\bfseries}

\setbeamertemplate{itemize item}[circle]
\setbeamertemplate{itemize subitem}[circle]

\setbeamercovered{transparent=4}

\setbeamertemplate{navigation symbols}{}

\setbeamertemplate{frametitle}
{
 \vspace{1ex}
 \insertframetitle\par
}

\setbeamertemplate{footline}[frame number]

\makeatletter
\newcommand*{\overlaynumber}{\number\beamer@slideinframe}
\makeatother


\setbeamertemplate{theorems}[numbered]

\AtBeginEnvironment{theorem}{%
  \setbeamercolor{block title}{fg=egyptianblue}
}
\AtBeginEnvironment{lemma}{%
  \setbeamercolor{block title}{fg=egyptianblue}
}
\AtBeginEnvironment{Proof}{%
  \setbeamercolor{block title}{fg=egyptianblue}
}


%---------------------------------------------------------


%!TEX root = IL.tex

\newcommand{\mi}[1]{\mathit{#1}}
\renewcommand{\vec}[1]{\bm{#1}}

\newcommand<>{\deemph}[1]{{\color#2{black!50}#1}}
%\newcommand<>{\empha}[1]{\textcolor{mediumelectricblue}{#1}}
\newcommand<>{\empha}[1]{{\color#2{trueblue}#1}}
\newcommand<>{\shade}[1]{{\color#2{black!10}#1}}

\newcommand<>{\df}[1]{\alert{\emph{#1}}}

\newcommand{\T}{\mathcal{T}}


%\newcommand{\alt}{\ensuremath{~\mid~}}
\newcommand{\akey}[1]{\textbf{#1}}
\newcommand{\seque}[1]{l_{\mathrm{#1}}}
\newcommand{\bool}{\akey{Bool}}


\newcommand{\nter}[2][]{{\textcolor{OliveGreen}{\ensuremath{\langle}\emph{\sffamily{#2}}\ensuremath{\rangle^{#1}}}}\xspace}

\newcommand{\expr}[1]{{\textcolor{mediumseagreen}{\texttt{#1}}}}
\newcommand{\ter}[1]{\expr{#1}\xspace}
\newcommand{\attr}[1]{\textbf{:#1}\xspace}
\newcommand{\kw}[1]{\textbf{#1}\xspace}

%\newcommand{\aattr}[1]{\textbf{#1}\xspace}
\newcommand{\comm}[1]{\textcolor{lighttaupe}{#1}}

\newcommand{\str}[1]{\ensuremath{\mathcal{#1}}}

\newcommand{\inputa}{\attr{input}}
\newcommand{\outputa}{\attr{output}}
\newcommand{\parama}{\attr{param}}
\newcommand{\inita}{\attr{init}}
\newcommand{\transa}{\attr{trans}}
\newcommand{\framea}{\attr{frame}}
\newcommand{\inva}{\attr{inv}}
\newcommand{\locala}{\attr{local}}
\newcommand{\subsysa}{\attr{subsys}}
\newcommand{\proca}{\attr{process}}
\newcommand{\assumptiona}{\attr{assumption}}
\newcommand{\reachablea}{\attr{reachable}}
\newcommand{\fairnessa}{\attr{fairness}}
\newcommand{\currenta}{\attr{current}}
\newcommand{\querya}{\attr{query}}
\newcommand{\queriesa}{\attr{queries}}
\newcommand{\modela}{\attr{model}}
\newcommand{\resulta}{\attr{result}}
\newcommand{\traila}{\attr{trail}}
\newcommand{\tracea}{\attr{trace}}
\newcommand{\prefixa}{\attr{prefix}}
\newcommand{\lassoa}{\attr{lasso}}

\newcommand{\definesys}{\kw{define-system}}
\newcommand{\checksys}{\kw{check-system}}
\newcommand{\checkres}{\kw{response}}
\newcommand{\sat}{\kw{sat}}
\newcommand{\unsat}{\kw{unsat}}









\title{
{\LARGE A Model Checking Intermediate Language}
\\[1ex]
}
\subtitle{\Large An Overview}
%
\author{\large The NSF:CCRI Team}

\date{FMCAD 2023}

%\institute{}

\begin{document}
%\maketitle

%---------------------------------------------------------------------

\begin{frame}
\titlepage

\end{frame}

%---------------------------------------------------------------------

\begin{frame}{Model Checking Intermediate Language (IL)  goals}
\large 

The IL has been designed to
\bigskip

\begin{itemize}
\item
be a \alert{general enough} intermediate \alert{target language} for MC
\medskip

\item
support a \alert{variety} of user-facing \alert{modeling languages}
\medskip

\item
be \alert{directly supported} by tools or \alert{compiled} to lower level languages
\medskip

\item
leverage SAT/SMT technology
\end{itemize}


\end{frame}

%---------------------------------------------------------------------

\begin{frame}{General design principles}
\large

IL models are meant to be \alert{produced and processed} mostly \\
\alert{by tools}
\bigskip
\medskip
\pause

So the IL was designed to have
\medskip

\setbeamercovered{transparent=12}

\begin{itemize}
\item<2,8> \alert{simple}, easily parsable \alert{syntax}
\item<3,8> a \alert{rich set} of \alert{data types}
\item<4,8> little syntactic sugar, at least initially
\item<5,8> \alert{well-understood semantics}
\item<6,8> a \alert{small} but comprehensive \alert{set of commands}
\item<7-> \alert{simple translations to lower level} languages \\
      such as Btor2 and Aiger
\end{itemize}

\setbeamercovered{transparent=4}

\end{frame}

%---------------------------------------------------------------------

\begin{frame}{Design principles --- implications }
\large

\begin{enumerate}
\item
\alert{Little direct support} for many of the \alert{features} offered by 
\medskip

\begin{itemize}\normalsize
\item
hardware modeling languages such as VHDL and Verilog or
\medskip

\item
system modeling languages such as SMV, TLA+, PROMELA, \\
UNITY, Lustre
\end{itemize}
\bigskip
\bigskip
\pause

\item
However, enough \alert{capability to support} the \alert{reduction} of problems
in those languages \alert{to} problems in the \alert{IL}
\end{enumerate}

\end{frame}

%---------------------------------------------------------------------

\begin{frame}{IL in a Nutshell}

\alert{Extension} the \alert{SMT-LIB} language with \alert{new commands} to define and check systems

\bigskip
\smallskip
\pause

Each \alert{system definition}
\medskip

\setbeamercovered{transparent=12}
\begin{itemize}
\item<2,8>
defines a \alert{transition system} via the use of SMT formulas
\medskip

\item<3,8>
generally imposes \alert{minimal syntactic restrictions} on those formulas
\medskip

\item<4,8>
is \empha{parametrized} by a \df{state signature}, a sequence of typed variables
\medskip

\item<5,8>
partitions state variables into \alert{input}, \alert{output} and \alert{local} variables
\medskip

\item<6,8>
is \alert{hierarchical}, i.e., may include (instances of) previously defined systems as subsystems 
\medskip

\item<7,8>
can \alert{encode} \alert{synchronous} \alert{and} (later) \alert{asynchronous} system composition
\end{itemize}
\setbeamercovered{transparent=4}

\end{frame}

%---------------------------------------------------------------------

\begin{frame}{IL in a Nutshell}

\alert{Extension} the \alert{SMT-LIB} language with \alert{new commands} to define and check systems

\bigskip
\smallskip
\pause

Each \alert{verification command}
\medskip

\setbeamercovered{transparent=12}
\begin{itemize}
\item<2,8>
refers to a \alert{previously defined system}
\medskip

\item<3,8>
queries the \emph{reachability} of a one-state formula
\medskip

\item<4,8>
can impose environmental assumptions on input
\medskip

\item<5,8>
can specifie fairness condition on queries
\medskip

\item<6,8>
is \alert{hierarchical}, i.e., may include (instances of) previously defined systems as subsystems 
\medskip

\item<7,8>
can \alert{encode} \alert{synchronous} \alert{and} (later) \alert{asynchronous} system composition
\end{itemize}
\setbeamercovered{transparent=4}

\end{frame}

%---------------------------------------------------------------------

\begin{frame}{Current focus}
\Large
\centering

\alert{Finite-state} systems
\bigskip
\bigskip
\pause

but with an eye to \alert{infinite-state} systems too

\end{frame}

%---------------------------------------------------------------------

%\begin{frame}{Preliminaries}
%
%Formally, a transition system is a pair $S$ of predicates of the form
%{\large
%\[
% S \ = \ (\,
% \lambda\vec i{:}\vec\delta\, \lambda\vec o{:}\vec\tau\: 
%    \alert{I_S}[\vec i,\vec o,\vec s],\;
% \lambda\vec i{:}\vec\delta\, \lambda\vec o{:}\vec\tau\,
% \lambda\vec{i'}{:}\vec\delta\, \lambda\vec{o'}{:}\vec\tau\:
%   \alert{T_S}[\vec i,\vec o,\vec s, \vec{i'},\vec{o'},\vec{s'}]\,
% )
%\]
%}
%where 
%\begin{itemize}
%\item 
%$\vec i$ and $\vec{i'}$ are two tuples of \df{input variables}, both of type $\vec\delta$
%\item 
%$\vec s$ and $\vec{s'}$ are two tuples of \df{local variables}, both of type $\vec\sigma$
%\item 
%$\vec o$ and $\vec{o'}$ are two tuples of \df{output variables}, both of type $\vec\tau$
%\item 
%$\alert{I_S}$, the \df{initial state condition} is an SMT-LIB formula with free vars from $[\vec i,\vec o,\vec s]$ 
%\item
%$\alert{T_S}$, the \df{transition condition} is an SMT-LIB formula with free vars from $[\vec i,\vec o,\vec s, \vec{i'},\vec{o'},\vec{s'}]$
%\end{itemize}
%
%\end{frame}

%---------------------------------------------------------------------

\begin{frame}{Technical preliminaries}

Formally, a transition system is a pair $S$ of predicates of the form
{\large
\[
 S \ = \ (\,
 \alert{I_S}[\vec i,\vec o,\vec s],\;
 \alert{T_S}[\vec i,\vec o,\vec s, \vec{i'},\vec{o'},\vec{s'}]\,
 )
\]
}
\pause

where 
\begin{itemize}
\item 
$\vec i$ and $\vec{i'}$ are two tuples of \df{input variables} with the same length and type
\item 
$\vec o$ and $\vec{o'}$ are two tuples of \df{output variables} with the same length and type
\item 
$\vec s$ and $\vec{s'}$ are two tuples of \df{local variables} with the same length and type
\item 
$\alert{I_S}$, the \df{initial state condition} is a formula with free vars from $[\vec i,\vec o,\vec s]$ 
\item
$\alert{T_S}$, the \df{transition condition} is a formula with free vars from $[\vec i,\vec o,\vec s, \vec{i'},\vec{o'},\vec{s'}]$
\end{itemize}
\bigskip

\pause
\textbf{Note:}
A (full) state of $S$ is a valuation of $(\vec i,\vec o,\vec s)$
\end{frame}

%---------------------------------------------------------------------

\begin{frame}[t]{SMT-LIB commands}
\vspace{5ex}

\only<1>{%
\textbf{\large As in SMT-LIB}
\bigskip

\expr{(\kw{set-logic} $L$)}
\medskip

\expr{(\kw{declare-sort} $s$ $n$)}
\medskip

\expr{(\kw{define-sort} $s$ ($u_1$ $\,\cdots$ $u_n$) $\tau$)}
\medskip

\expr{(\kw{declare-fun} $f$ (($x_1$ $\sigma_1$) $\,\cdots$ ($x_n$ $\sigma_n$)) $\sigma$)}
\medskip

\expr{(\kw{define-fun} $f$ (($x_1$ $\sigma_1$) $\,\cdots$ ($x_n$ $\sigma_n$)) $\sigma$ $t$)}
\medskip

\expr{(\kw{declare-datatype} $d$ ($\,\cdots$))}
\medskip

\expr{(\kw{assert} $F$)}
\medskip

{\small (perhaps a few more)}
}

\only<2->{%
\textbf{\large New}
\bigskip

\expr{(\definesys $S$ $\,\cdots$)}
\medskip

\expr{(\checksys $S$ $\,\cdots$)}
\vspace{4ex}

\expr{(\kw{declare-enum-sort} $S$ ($c_1$ $\,\cdots$ $c_n$))}
\medskip

\expr{(\kw{declare-range-sort} $S$ ($m$ $n$))}
}
\vspace{5ex}

\only<3->{%
\textbf{Note:}
IL inherits SMT-LIB's concrete syntax, based on s-expressions
\\
\phantom{\textbf{Note:}}
(only prefix operators)

}
\end{frame}

%---------------------------------------------------------------------

\begin{frame}{Logical semantics}

A \expr{\definesys} command implicitly defines 
a \alert{\emph{model}} (i.e., a Kripke structure) 
of \alert{First-Order} Linear Temporal Logic (FO-LTL)
\bigskip
\pause

An FO-LTL formula $F[\vec f, \vec x, \vec x']$ with 
\smallskip

\begin{itemize}
\item free (immutable) constants/functions (aka, uninterpreted symbols) from $\vec f$ 
\item free (mutable) variables from $\vec x, \vec x'$ 
\end{itemize}
\smallskip

is \df{satisfiable} in an SMT theory $\T$ if there is
\smallskip
\pause

\begin{enumerate}
\item a $\T$-interpretation $\str I$ of $\vec f$ and 
\item an infinite trace $\pi$ over $\vec x$ in $\str I$
\end{enumerate}
\smallskip

that \alert{satisfy} $F$

\end{frame}

%---------------------------------------------------------------------

\begin{frame}{\only<3->{\alert{Finite-}}Trace semantics}
\small

Fix
\begin{itemize}
\item an FOL-LTL formula $F[\vec f, \vec x, \vec x']$ over a theory $\T$
\item a $\T$-interpretation $\str I$ of $\vec f$
\item an infinite trace $\pi = s_0, s_1, \ldots$ \ 
where $s_i$ is an assignment of $\vec x$ into $\str I$ \ for all $i \geq 0$   
\end{itemize}
\bigskip

Let $\pi^i = s_i, s_{i+1}, \ldots$ for all $i \geq 0$
\bigskip
\pause

$(\str I, \pi)$ \ \df{\only<3->{$n$-}satisfies $F$} \ \only<3->{for some $n > 0$},
written $(\str I, \pi) \models\only<3->{_n} F$, iff
\medskip

\begin{tabular}{lll}
1. & ${\str I}[\vec x \mapsto s_0(\vec x),\, \vec x' \mapsto s_1(\vec x)]$
     \ satisfies $F$ 
   & when $F$ is atomic
\\[.5ex]
2. & $({\str I}, \pi) \not\models\only<3->{_n} G$
   & when $F$ is $\lnot G$
\\[.5ex]
3. & $({\str I}, \pi) \models\only<3->{_n} G_j$ \ for $j=1,2$
   & when $F$ is $G_1 \land G_2$
\\[.5ex]
4. & $({\str I}, \pi^1) \models\only<3->{_{n-1}} G$ \only<3->{and $n - 1 > 0$}
   & when $F$ is $\textbf{next}\ G$
\\[.5ex]
5. & $({\str I}, \pi^i) \models\only<3->{_{n-i}} G$ \ for all $i = 0, \ldots, \only<3->{n-1}$
   & when $F$ is $\textbf{always}\ G$
\\[.5ex]
6. & $({\str I}, \pi^i) \models\only<3->{_{n-i}} G$ \ for some $i = 0, \ldots, \only<3->{n-1}$
  & when $F$ is $\textbf{eventually}\ G$
\\[.5ex]
7. & \ldots
\end{tabular}

\end{frame}

%---------------------------------------------------------------------

\newcommand{\definesystem}{%
\expr{(\definesys $S$ \\
\begin{tabular}{l@{~}l@{\;}l}
 \inputa 
   & (\,($i_1$ $\mi{\delta}_1$) $\,\cdots$ ($i_m$ $\mi{\delta}_m$)\,)
   & \comm{; input vars} \\
 \outputa
   & (\,($o_1$ $\mi{\tau}_1$) $\,\cdots$ ($o_n$ $\mi{\tau}_n$)\,)
   & \comm{; output vars} \\
 \locala 
   & (\,($s_1$ $\mi{\sigma}_1$) $\,\cdots$ ($s_p$ $\mi{\delta}_p$)\,)
   & \comm{; local vars} \\
 \inita 
   & $I$
   & \comm{; initial state formula} \\
 \transa 
   & $T$
   & \comm{; transition formula} \\
 \inva 
   & $P$
   & \comm{; invariant formula} \\
% ~~$\vdots$
%   & $\vdots$ & \hspace{5em} $\vdots$ \\
% \inva 
%   & $P_n$
%   & \comm{; invariant formula} \\
\end{tabular} \\
)
}}

%---------------------------------------------------------------------

\begin{frame}{}
\LARGE
\centering
\bigskip

\textbf{Model Specification}

\end{frame}

%---------------------------------------------------------------------

\begin{frame}[t]{Atomic system \alert{definition}}

{\large \definesystem}
\medskip

\only<2>{
where

\begin{itemize}
\item
each var gets a \alert{primed copy}: $i'_1$, \ldots, $o'_1$, \ldots, $s'_1$, \ldots

\item
$I$ and $P$ are \alert{one-state} formulas (over unprimed vars only)

\item
$T$ is a \alert{two-state} formula (over unprimed and primed vars)

\item
all \alert{attributes} are \alert{optional} and
their \alert{order} is \alert{immaterial}

\begin{itemize}
\item[${\color{black}\blacktriangleright}$]
however, \inputa, \outputa, \locala must occur before
\inita, \transa, \inva 
\end{itemize}
\end{itemize}
}

\end{frame}

%---------------------------------------------------------------------

\begin{frame}{Default values for missing attributes}
\centering
\large

\begin{tabular}{ll}
 \hline
 {\bf attribute} & {\bf default} \\
 \hline
 \expr{\inputa} & \expr{()} \\
 \expr{\outputa} & \expr{()} \\
 \expr{\locala}  & \expr{()} \\
 \expr{\inita} & \expr{true} \\
 \expr{\transa}  & \expr{true} \\
 \expr{\inva}  & \expr{true}
\end{tabular}

\end{frame}

%---------------------------------------------------------------------

\begin{frame}[t]{Examples}

\expr{\normalsize
\comm{%
; The output of Delay is initially in [0,10] and  \\
; then is the previous input \\
;} \\
(\definesys Delay \inputa ((in Int)) \outputa ((out Int)) \\
~~\inita (<= 0 out 10) \\
~~\transa (= out' in) \\
)}
\bigskip

\only<2->{
\textbf{Example trace:}
\medskip

\centering

\begin{tabular}{r|rrrrrrrrr}
step & 0 & 1 & 2 & 3 & 4 & 5 & 6 & \ldots \\
\hline
\expr{in}  & 1 & 2 & 3 & 4 & 3 & 2 & 1 & \ldots \\
\expr{out} & 9 & 1 & 2 & 3 & 4 & 3 & 2 & \ldots 
\end{tabular}
}

\end{frame}

%---------------------------------------------------------------------

\begin{frame}[t]{Examples}

\expr{%
\comm{%
; A clocked lossless channel that stutters when clock is false \\
;} \\
(\definesys ClockedChannel  \\
~~\inputa ((clock Bool) (in Int)) \\
~~\outputa ((out Int)) \\
~~~~~~~~\comm{; out is unconstrained when clock is false} \\
~~\inita (=> clock (= out in))  \\
~~\transa (ite clock' (= out' in') (= out' out)) \\
)}
\bigskip

\only<2->{
\textbf{Example trace:}
\smallskip

\centering

\begin{tabular}{r|rrrrrrrrr}
step & 0 & 1 & 2 & 3 & 4 & 5 & 6 & \ldots \\
\hline
\expr{clock}  & F & T & T & F & F & T & F & \ldots \\
\expr{in}     & 1 & 2 & 3 & 4 & 5 & 6 & 7 & \ldots \\
\expr{out}   & -3 & 2 & 3 & 3 & 3 & 6 & 6 & \ldots 
\end{tabular}
}

\end{frame}

%---------------------------------------------------------------------

\begin{frame}[t]{Example: timed light switch}

\only<1>{
\expr{TimedSwitch} models a timed light switch 
\smallskip

where, once on,
the light stays on for 10 steps unless it is switched off before
\bigskip

A Boolean input is provided as a toggle signal 
}

\small
\only<2->{
\expr{%
(\kw{define-enum-sort} LightStatus (On Off)) \\[1.5ex]
}}

%
\only<2>{
\expr{%
\comm{; Guarded-transitions-style definition} \\
(\definesys TimedSwitch
  \inputa ((press Bool)) \outputa ((sig Bool)) \\
~~\locala ((s LightStatus) (n Int)) \\
~~\inva (= sig (= s On)) \\
~~\inita (and (= n 0) (ite press (= s On) (= s Off))) \\
~~\transa (and \\
~~~~(=> (and (= s~ Off) (not press'))~~~~~~~~~~~\comm{; Off ->} \\
~~~~~~~~(and (= s' Off) (= n' n)))~~~~~~~~~~~~~~\comm{; Off} \\
~~~~(=> (and (= s~ Off) press')~~~~~~~~~~~~~~~~~\comm{; Off ->} \\
~~~~~~~~(and (= s' On)~ (= n' n)))~~~~~~~~~~~~~~\comm{; On} \\
~~~~(=> (and (= s~ On)~ (not press') (< 10 n))~~\comm{; On ->} \\
~~~~~~~~(and (= s' On)~ (= n' (+ n 1))))~~~~~~~~\comm{; On} \\
~~~~(=> (and (= s~ On)~ (or press' (>= n 10))~~~\comm{; On ->} \\
~~~~~~~~(and (= s' Off) (= n' 0)))~~~~~~~~~~~~~~\comm{; Off} \\
~~) \\
)
}}

\only<3>{
\expr{%
\comm{; Set-of-transitions-style definition} \\
(\definesys TimedSwitch2
  \inputa ((press Bool)) \outputa ((sig Bool)) \\
~~\locala ((s LightStatus) (n Int)) \\
~~\inva (= sig (= s On)) \\
~~\inita (and (= n 0) (ite press (= s On) (= s Off))) \\
~~\transa \\
~~~~(\kw{let} (\comm{; Transitions} \\
~~~~~~~~~~(stay-off (and (= s Off) (not press') (= s' Off) (= n' n)))  \\
~~~~~~~~~~(turn-on~ (and (= s Off) press' (= s' On) (= n' n))) \\
~~~~~~~~~~(stay-on~ (and (= s On) (not press') (< n 10) \\
~~~~~~~~~~~~~~~~~~~~~~~~~(= s' On) (= n' (+ n 1)))) \\
~~~~~~~~~~(turn-off (and (= s On) (or press' (>= n 10)) \\
~~~~~~~~~~~~~~~~~~~~~~~~~(= s' Off) (= n' 0)))  \\
~~~~~~~~~) \\
~~~~~~(or stay-off turn-on turn-off stay-on) \\
~~~~) \\
)
}}


\only<4>{
\expr{%
\comm{; Equational-style definition} \\
(\definesys TimedSwitch3 
  \inputa ((press Bool)) \outputa ((sig Bool)) \\
~~\locala ((s LightStatus) (n Int)) \\
~~\inva (= sig (= s On)) \\
~~\inita (and (= n 0) (ite press (= s On) (= s Off))) \\
~~\transa (and \\
~~~~(= s' (ite press' (flip s) \\
~~~~~~~~~~~~(ite (or (= s Off) (>= n 10)) Off \\
~~~~~~~~~~~~~~On))) \\
~~~~(= n' (ite (or (= s Off) (s' Off)) 0 \\
~~~~~~~~~~~~(+ n 1))) \\
~~) \\
) \\
\ \\
(\kw{define-fun} flip ((s LightStatus)) LightStatus \\
~~(ite (= s Off) On Off) \\
) \\
}}

\end{frame}

%---------------------------------------------------------------------

\begin{frame}[t]{Examples}
\small

\visible<1->{%
\expr{%
(\kw{declare-datatype} Event (par (X) (Abs) (Pres (val X))))
}
\medskip

\begin{tabular}{ll}
Values of sort \expr{(Event Int)}:& \expr{Abs}, \expr{(Pres 12)}, \expr{(Pres -23)}, \ldots
\\
Values of sort \expr{(Event Bool)}:& \expr{Abs}, \expr{(Pres true)}, \expr{(Pres false)}
\end{tabular}
\bigskip

\visible<2->{
\expr{%
\comm{%
; An event-triggered channel that arbitrarily loses its input data} \\
(\definesys LossyIntChannel  \\
~~\inputa ((in (Event Int))) \\
~~\outputa ((out (Event Int))) \\
~~\inva (or (= out in) (= out Abs)) \\
) \\
}}
}
\bigskip

\visible<3->{%
\expr{%
\comm{%
; Equivalent formulation using unconstrained local state} \\
(\definesys LossyIntChannel  \\
~~\inputa ((in (Event Int))) \\
~~\outputa ((out (Event Int))) \\
~~\locala ((s Bool)) \\
~~\comm{; input event is relayed or not depending on value of s} \\
~~\inva (= out (ite s in Abs)) \\
) \\
}}

\end{frame}

%---------------------------------------------------------------------

\begin{frame}[t]{Atomic system \alert{definition} --- Semantics}

\definesystem
%
\only<2-3>{
{\large
\[
 S = (\alert{I_S}, \alert{T_S})
   = (I[\vec i,\vec o,\vec s], \ 
      P[\vec i,\vec o,\vec s] \land
      T[\vec i,\vec o,\vec s,\vec{i'},\vec{o'},\vec{s'}]
     )
\]}

where \ 
$\vec i = (i_1,\ldots,i_m)$, 
$\vec o = (o_1,\ldots,o_n)$, 
$\vec s = (s_1,\ldots,s_n)$
\bigskip
}

\visible<3>{\large
\emph{$S$ denotes the set of all \alert{infinite} traces 
that satisfy the \alert{FO-LTL} formula}
\[
  \alert{I_S}%[\vec i,\vec o,\vec s] 
  ~\land~
  \textbf{always}\ \alert{T_S}%[\vec i,\vec o,\vec s,\vec{i'},\vec{o'},\vec{s'}]
\]}%
\only<4->{\normalsize
\textbf{Note:}
\smallskip

Systems are meant to be \df{progressive}: every reachable state has a successor wrt $T_S$
\smallskip

However, they may not be because of the generality of $T$ and $P$
\smallskip

(It is possible to define deadlocking systems)
}
\end{frame}

%---------------------------------------------------------------------

\begin{frame}[t]{Composite system \alert{definition}}

\expr{(\definesys $S$ \\
\begin{tabular}{l@{~}l@{\;}l}
 \inputa 
   & (\,($i_1$ $\mi{\delta}_1$) $\,\cdots$ ($i_m$ $\mi{\delta}_m$)\,)
   & \comm{; input vars} \\
 \outputa
   & (\,($o_1$ $\mi{\tau}_1$) $\,\cdots$ ($o_n$ $\mi{\tau}_n$)\,)
   & \comm{; output vars} \\
 \locala 
   & (\,($s_1$ $\mi{\sigma}_1$) $\,\cdots$ ($s_p$ $\mi{\delta}_p$)\,)
   & \comm{; local vars} \\
 \subsysa 
   & (\,$N_1$ ($S_1$ $\vec x_1$ $\vec y_1$)\,)
   & \comm{; component subsystem} \\
 $\quad\cdots$ & $\quad\cdots$ & $\quad\cdots$
 \\
 \subsysa 
   & (\,$N_q$ ($S_q$ $\vec x_q$ $\vec y_q$)\,)
   & \comm{; component subsystem}
\end{tabular} \\
)
}

\small
\medskip

\only<2->{
where

\begin{enumerate}
\item $q > 0$ and each $S_i$ is the name of a system other than $S$
\item $S_1, \ldots, S_q$ need not be all distinct
\item each $N_i$ is a local synonym for $S_i$, with $N_1$, \ldots $N_q$ distinct
\item each $\vec x_i$ consists of $S$'s variables of the same type as $S_i$'s input
\item each $\vec y_i$ consists of $S$'s local/output variables of the same type as $S_i$'s output
\item the directed subsystem graph rooted at $S$ is acyclic 
\end{enumerate}
}

\end{frame}

%---------------------------------------------------------------------

\begin{frame}[t]{Composite system \alert{definition} extended}

\expr{(\definesys $S$ \\
\begin{tabular}{l@{~}l@{\;}l}
 \inputa 
   & (\,($i_1$ $\mi{\delta}_1$) $\,\cdots$ ($i_m$ $\mi{\delta}_m$)\,)
   & \comm{; input vars} \\
 \outputa
   & (\,($o_1$ $\mi{\tau}_1$) $\,\cdots$ ($o_n$ $\mi{\tau}_n$)\,)
   & \comm{; output vars} \\
 \locala 
   & (\,($s_1$ $\mi{\sigma}_1$) $\,\cdots$ ($s_p$ $\mi{\delta}_p$)\,)
   & \comm{; local vars} \\
 \subsysa 
   & (\,$N_1$ ($S_1$ $\vec x_1$ $\vec y_1$)\,)
   & \comm{; component subsystem} \\
 $\quad\cdots$ & $\quad\cdots$ & $\quad\cdots$
 \\
 \subsysa 
   & (\,$N_q$ ($S_q$ $\vec x_q$ $\vec y_q$)\,)
   & \comm{; component subsystem} \\
 \inita 
   & $I$
   & \comm{; initial state formula} \\
 \transa 
   & $T$
   & \comm{; transition formula} \\
 \inva 
   & $P$
   & \comm{; invariant formula} \\
\end{tabular} \\
)
}
\vspace{3ex}

\visible<2->{
\begin{center}
\large\em Composition is \alert{synchronous} by default
\end{center}
}

\end{frame}

%---------------------------------------------------------------------

%\begin{frame}[t]{Examples}
%\footnotesize
%
%\expr{%
%;~~~~~~~~+------------------------------------------------------------+ \\
%;~~~~~~~~|~~~~~~~~~~~~~~~~~~~~~~~~~~~~~~~~~~~~~~~~~~~~~~~~~~~~~~~~~~~~| \\
%;~~~~~~~~|~+--------------------------------------------------------+~| \\
%;~~~~~~~~|~|~~~~~~~~~~~~~~~~~~~~~~~~~~~~~~~~~~~~~~~~~~~~~~+-------+~|~| \\
%;~~~~~~~~+-|-----------------------------------|``-.~~set |~~~~~~~|~|~| \\
%;~~~~~~~~|~|~~~~~~~~~~~~~~~~~~~~~~~~~~~~~~~~~~~~|`-.\_~~~~|~~~~:---->|~~~~|~~|~~| \\
%;~~~~~~~~| +->|``-.~~~~~~~~~~~~~+--|~~~\_]o--|..-`~~~~~~| Latch~|~|~| \\
%;~~~~~~~~|~~~~|~~~~:--+----\``-.~~~|~~|.-`~~~~~~~~~~reset |~~~~~~~|-+-+--> out \\
%;~~~inc -+----|..-`~~~|~~~~~)~~~:--+--------------------->|~~~~~~~|~~~| \\
%;~~~~~~~~|~~~~~~~~~~~~| +--/..-`~~~~~~~~~~~~~~~~~~~~~~~~~~+-------+~~~| \\
%;~~~~~~~~|~~~~~~~~~~~~|~~~~~~~~~~~~~~~~~~~~~~~~~~~~~~~~~~~~~~~~~~~~~~~| \\
%;~start~----------------+~~~~~~~~~~~~~~~~OneBitCounter~~~~~~~~~~~~~~~~| \\
%;~~~~~~~~|~~~~~~~~~~~~|~~~~~~~~~~~~~~~~~~~~~~~~~~~~~~~~~~~~~~~~~~~~~~~| \\
%;~~~~~~~~+------------+-----------------------------------------------+ \\
%;~~~~~~~~~~~~~~~~~~~~~|     \\
%;~~~~~~~~~~~~~~~~~~~~~v carry
%}
%
%\end{frame}

%---------------------------------------------------------------------
\begin{frame}[t]{Examples}

\expr{%
\comm{; One-step delay} \\
(\definesys Delay \inputa ((i Int)) \outputa ((o Int)) \\
~~\locala ((s Int)) \\
~~\inva (= s i) 
~~\inita (= o 0)
~~\transa (= o' s) \\
)\\
\comm{; Two-step delay} \\
(\definesys Delay2 \inputa ((in Int)) \outputa ((out Int)) \\
~~\locala ((temp Int)) \\
~~\subsysa (D1 (\alert{Delay} in temp)) \\
~~\subsysa (D2 (\alert{Delay} temp out)) \\
)}
\medskip

\only<2->{
\textbf{Example trace:}
\smallskip

\centering

\begin{tabular}{r|rrrrrrrrr}
step        & 0 & 1 & 2 & 3 & 4 & 5 & 6 & 7 & \ldots \\
\hline
\expr{in}   & 2 & 3 & 4 & 5 & 6 & 7 & 8 & 9 & \ldots \\
\expr{temp} & 0 & 2 & 3 & 4 & 5 & 6 & 7 & 8 &\ldots \\
\expr{out}  & 0 & 0 & 2 & 3 & 4 & 5 & 6 & 7 & \ldots 
\end{tabular}
}

\end{frame}

%---------------------------------------------------------------------

\begin{frame}[t]{Examples}
\footnotesize

\expr{%
(\definesys Latch
  \inputa ((s Bool) (r Bool))
  \outputa ((o Bool)) \\
~~\locala ((b Bool)) \\
~~\transa (= o' (or (and s (or (not r) b)) \\
~~~~~~~~~~~~~~~~~~~(and (not s) (not r) o))) \\
)
\\[1.3ex]
(\definesys OneBitCounter
  \inputa ((inc Bool) (start Bool)) \\
~~\outputa ((out Bool) (carry Bool)) \\
~~\locala ((set Bool) (reset Bool)) \\
~~\subsysa (L (\alert{Latch} set reset out)) \\
~~\inva (and (= set (and inc (not reset))) \\
~~~~~~~~~~~~(= reset (or carry start)) \\
~~~~~~~~~~~~(= carry (and inc out))) \\
)
\\[1.3ex]
(\definesys ThreeBitCounter \\
~~\inputa ((inc Bool) (start Bool)) \\
~~\outputa ((out0 Bool) (out1 Bool) (out2 Bool)) \\
~~\locala ((car0 Bool) (car1 Bool) (car2 Bool)) \\
~~\subsysa (C1 (\alert{OneBitCounter} inc start out0 car0)) \\
~~\subsysa (C2 (\alert{OneBitCounter} car0 start out1 car1)) \\
~~\subsysa (C3 (\alert{OneBitCounter} car1 start out2 car2)) \\
)
}
\end{frame}

%---------------------------------------------------------------------

\begin{frame}[t]{Composite system \alert{definition} --- Semantics}

\expr{(\definesys $S$ \\
\begin{tabular}{l@{~}l@{\;}l}
 \inputa 
   & (\,($i_1$ $\mi{\delta}_1$) $\,\cdots$ ($i_m$ $\mi{\delta}_m$)\,)
   & \comm{; input vars} \\
 \outputa
   & (\,($o_1$ $\mi{\tau}_1$) $\,\cdots$ ($o_n$ $\mi{\tau}_n$)\,)
   & \comm{; output vars} \\
 \locala 
   & (\,($s_1$ $\mi{\sigma}_1$) $\,\cdots$ ($s_p$ $\mi{\delta}_p$)\,)
   & \comm{; local vars} \\
 \subsysa 
   & (\,$N_1$ ($S_1$ $\vec x_1$ $\vec y_1$)\,)
   & \comm{; component subsystem} \\
 $\quad\cdots$ & $\quad\cdots$ & $\quad\cdots$
 \\
 \subsysa 
   & (\,$N_q$ ($S_q$ $\vec x_q$ $\vec y_q$)\,)
   & \comm{; component subsystem}
\end{tabular} \\
)
}
\bigskip

\small

\only<2>{
Let \ 
\(
 S_k = (\,
   \alert{I_k}[\vec i_k,\vec o_k,\vec s_k],\;
   \alert{T_k}[\vec i_k,\vec o_k,\vec s_k,\vec{i'}_k,\vec{o'}_k,\vec{s'}_k]
   \,)
\)
for $k = 1, \ldots, q$,
with $\vec s_1, \ldots, \vec s_q$ all distinct
\smallskip

Let \ 
$\vec i = (i_1, \ldots, i_m)$, \ 
$\vec o = (o_1, \ldots, o_n)$, \ 
$\vec s = s_1, \ldots, s_q, \vec s_1, \ldots, \vec s_q$

{\normalsize
\[
S = (\alert{I_S}[\vec i,\vec o,\vec s], 
     \alert{T_S}[\vec i,\vec o,\vec s,\vec{i'},\vec{o'},\vec{s'}])
\]

\begin{tabular}{l@{\qquad}l@{\qquad\quad}l}
 with
 &
 $\alert{I_S} ~=~ 
 \bigwedge_{k=1}^q \alert{I_k}[\vec x_k,\vec y_k,\vec s_k]$
 &
 $\alert{T_S} ~=~ 
 \bigwedge_{k=1}^q \alert{T_k}[\vec x_k,\vec y_k,\vec s_k,\vec{x'}_k,\vec{y'}_k,\vec{s'}_k]$
\end{tabular}
}}

\end{frame}

%---------------------------------------------------------------------

\begin{frame}[t]{Composite system \alert{definition} extended --- Semantics}

\expr{(\definesys $S$ \\
\begin{tabular}{l@{~}l@{\;}l}
 \inputa 
   & (\,($i_1$ $\mi{\delta}_1$) $\,\cdots$ ($i_m$ $\mi{\delta}_m$)\,)
   & \comm{; input vars} \\
 \outputa
   & (\,($o_1$ $\mi{\tau}_1$) $\,\cdots$ ($o_n$ $\mi{\tau}_n$)\,)
   & \comm{; output vars} \\
 \locala 
   & (\,($s_1$ $\mi{\sigma}_1$) $\,\cdots$ ($s_p$ $\mi{\delta}_p$)\,)
   & \comm{; local vars} \\
 \subsysa 
   & (\,$N_1$ ($S_1$ $\vec x_1$ $\vec y_1$)\,)
   & \comm{; component subsystem} \\
 $\quad\cdots$ & $\quad\cdots$ & $\quad\cdots$
 \\
 \subsysa 
   & (\,$N_q$ ($S_q$ $\vec x_q$ $\vec y_q$)\,)
   & \comm{; component subsystem} \\
 \inita 
   & $I$
   & \comm{; initial state formula} \\
 \transa 
   & $T$
   & \comm{; transition formula} \\
 \inva 
   & $P$
   & \comm{; invariant formula} \\
\end{tabular} \\
)
}

\[
S = (\alert{I_S}[\vec i,\vec o,\vec s], 
     \alert{T_S}[\vec i,\vec o,\vec s,\vec{i'},\vec{o'},\vec{s'}])
\]

\begin{tabular}{l@{\qquad}l@{\qquad}l}
 with
 &
 $\alert{I_S} ~=~ 
 \alert{I} \land 
 \bigwedge_{k=1}^q \alert{I_k}[\vec x_k,\vec y_k,\vec s_k]$
 &
 $\alert{T_S} ~=~ 
 \alert{P} \land \alert{T} \land 
 \bigwedge_{k=1}^q \alert{T_k}[\vec x_k,\vec y_k,\vec s_k,\vec{x'}_k,\vec{y'}_k,\vec{s'}_k]$
\end{tabular}

\end{frame}

%---------------------------------------------------------------------

\begin{frame}[t]{Expressiveness}

\alert{\definesys + SMT-LIB commands and types} appear sufficient

to allow faithful reductions from (full or large fragment of)
\medskip

\begin{itemize}
\item Moore and Mealy machines
\item I/O automata
\item SMV and nuXMV
\item UNITY
\item TLA+
\item Reactive Modules
\item Lustre
\item SAL
\end{itemize}

\end{frame}

%---------------------------------------------------------------------

\begin{frame}{}
\LARGE
\centering
\bigskip

\textbf{Model Checking}

\end{frame}

%---------------------------------------------------------------------

\begin{frame}[t]{System \alert{checking} command \only<4->{--- Semantics}}

\expr{%
(\checksys $S$ \\
\begin{tabular}{l@{~}l@{\;}l}
 \inputa 
   & (\,($i_1$ $\mi{\delta}_1$) $\,\cdots$ ($i_m$ $\mi{\delta}_m$)\,)
   & \comm{; renaming of S's input vars} \\
 \outputa
   & (\,($o_1$ $\mi{\tau}_1$) $\,\cdots$ ($o_n$ $\mi{\tau}_n$)\,)
   & \comm{; renaming of S's output vars} \\
 \locala 
   & (\,($s_1$ $\mi{\sigma}_1$) $\,\cdots$ ($s_p$ $\mi{\delta}_p$)\,)
   & \comm{; renaming of S's local vars} \\
 \assumptiona
   & ($a$ $A$)
   & \comm{; environmental assumption} \\
 \reachablea 
   & ($r$ $R$)
   & \comm{; reachability condition} \\
 \fairnessa 
   & ($f$ $F$)
   & \comm{; fairness condition} \\
 \currenta 
   & ($c$ $C$)
   & \comm{; initiality condition } \\
 \only<1-3,7->{%
 \querya 
   & ($q$ ($g_1$ $\,\cdots$ $g_q$))
   & \comm{; trace query to be checked }
 }%
 \only<4>{%
   \alert{\querya} & ($q$ ($a$ $r$))
 }%
 \only<5>{%
   \alert{\querya} & ($q$ ($a$ $f$ $r$))
 }%
 \only<6>{%
   \alert{\querya} & ($q$ ($a$ $c$ $r$))
 }%
% \\
% \only<7>{%
%   \alert{\queriesa} & (($q_1$ ($g_{1,1}$ $\,\cdots$ $g_{1,q_1}$)) $\cdots$ ($q_k$ ($g_{k,1}$ $\,\cdots$ $g_{k,q_k}$)))
% }%
\end{tabular} \\
)
}
\medskip


\only<2>{\small
where
\begin{itemize}
\item $a$, $r$, $f$, $c$, $q$\: are \alert{identifiers}; 
 each $g_i$ ranges over $\{a,\,r,\,f,\,c\}$
\item $C$\: is a \alert{one-state (non-temporal) formula} over the given vars
\item $A$, $R$, $F$\: are \alert{one- or two-state (non-temporal) formulas} over the given vars
\item
all \alert{attributes} are \alert{optional} and their \alert{order} is \alert{immaterial}
\item
all attributes but the first three are \alert{repeatable}
\end{itemize}
}


%\only<4>{
%\medskip
%\textbf{Notation}
%\medskip
%
%\(\begin{array}{r}
% \textbf{always}_n\ P \\[.6ex]
% \textbf{eventually}_n\ P
%\end{array}
%\)
%are satisfiable if they are $n$-satisfied by some 
%}

\only<4-6>{
\smallskip
Query $q$ \alert{succeeds} iff the formula below is \alert{\only<4,6>{$\alert{n}$-}satisfiable} in LTL%
\only<4,6>{ \alert{for some} $n > 0$}
\[
 \only<6>{C} 
 \only<1-5>{I_S}
 \only<1-6>{\land \textbf{always}\ T_S}
 \land \textbf{always}\ A
 \land \textbf{eventually}\ R
 \uncover<5>{\land \textbf{always}\ \textbf{eventually}\ F}
\]
\small
where
$I_S$ and $T_S$ are the initial state and transition predicates of $S$
\alert{modulo} the renamings above 
}
\only<7->{
\medskip

For each successful query, 
%\smallskip
%
%\quad 
the model checker is expected to produce
\begin{itemize}
\item a \alert{$\T$-interpretation} $\str I$ (of the free immutable symbols) and
\item a \df{witnessing} trace in $\str I$
\end{itemize}
\medskip
}

\only<8->{
\centering
\em Different queries may be given different interpretations and traces
}
\end{frame}

%---------------------------------------------------------------------

\begin{frame}{Example --- Non-deterministic arbiter}

\expr{%
(\definesys NonDetArbiter \\
~~\inputa~~((r1 Bool) (r2 Bool)) \\
~~\outputa ((g1 Bool) (g2 Bool)) \\
~~\locala ((s Bool)) \\
~~\inva (and \\
~~~~(=> (and (not r1) (not r2)) \\
~~~~~~~~(and (not g1) (not g2))) \\
~~~~(=> (and r1 (not r2)) \\
~~~~~~~~(and g1 (not g2))) \\
~~~~(=> (and (not r1) r2) \\
~~~~~~~~(and (not g1) g2)) \\
~~~~(=> (and r1 r2) \\
~~~~\comm{; unconstrained value of s used as non-deterministic choice} \\
~~~~~~~~(ite s (and g1~~~~~~~(not g2)) \\
~~~~~~~~~~~~~~~(and (not g1) g2~~~~~))) \\
~~) \\
)
}
\end{frame}

%---------------------------------------------------------------------

\begin{frame}{Example --- Non-deterministic arbiter}

\expr{%
(\checksys NonDetArbiter \\
~~\inputa ((req1 Bool) (req2 Bool)) \\
~~\outputa ((gr1 Bool) (gr2 Bool)) \\[1ex]
~~\comm{; There are never concurrent requests} \\
~~\assumptiona (a1 
      (not (and req1 req2))) \\[1ex]
~~\comm{; The same request is never issued twice in a row} \\
~~\assumptiona (a2 (and (=> req1 (not req1')) \\
~~~~~~~~~~~~~~~~~~~~~~~(=> req2 (not req2')))) \\[1ex]
~~\comm{; Neg of:~Every request is immediately granted} \\
~~\reachablea (r (not (and (=> req1 gr1) (=> req2 gr2)))) \\[1ex]
~~\comm{; check the reachability of r under assumptions a1 and a2} \\
~~\querya (q (a1 a2 r)) \\
)
}
\end{frame}

%---------------------------------------------------------------------

\begin{frame}{Example --- Temporal queries}

\scriptsize
\expr{%
(\definesys Historically \inputa ((b Bool)) \outputa ((hb Bool)) \\
~~\inita (= hb b)  \transa (= hb' (and b' hb))) 
\\[1.4ex]
(\definesys Before \inputa ((b Bool)) \outputa ((bb Bool)) \\
~~\inita (= bb false) \transa (= bb' b)) 
\\[1.4ex]
(\definesys Count \inputa ((b Bool)) \outputa ((c Int)) \\
~~\inita (= c (ite b 1 0)) \transa (= c' (+ c (ite b' 0 1)))) 
\\[1.4ex]
(\definesys Monitor \inputa ((r1 Bool) (r2 Bool))
  \outputa ((g1 Bool) (g2 Bool)) \\
~~\locala ((a1 Bool) (a2 Bool) (b0 Bool) (b1 Bool) (b2 Bool) \\
~~~~~~~~~  (h1 Bool) (h2 Bool) (c Int) (bf Bool)) \\
~~\subsysa (NDA (NonDetArbiter r1 r2 g1 g2)) \\
~~\subsysa (His1 (Historically a1 h1)) \\
~~\subsysa (His2 (Historically a2 h2)) \\
~~\subsysa (Cnt (Count g1 c)) \\
~~\subsysa (Bf (Before b0 bf)) \\
~~\inva (and \\
~~~~~\comm{; a1 = no concurrent requests~~~~~~~a2 = no concurrent grants} \\
~~~~(= a1 (and (not r1) (not r2))) ~~ (= a2 (and (not g1) (not g2))) ~~ (= b0 (= c 4)) \\
~~~~(= b1 (=> h1 h2)) \comm{; b1 = if there have been no requests, there have been no grants}\\
~~~~(= b2 (=> bf (not g1))))) \comm{; b2 = a request is granted at most 4 times}
\\[1.4ex]
(\checksys Monitor
  \inputa ((r1 Bool) (r2 Bool)) \\
~~\outputa ((g1 Bool) (g2 Bool)) \\
~~\locala (\_ \_ \_ (b1 Bool) (b2 Bool) \_ \_ \_ \_) \\
~~\assumptiona (A (not (and r1 r2)))
  \reachablea (R (not (and b1 b2))) 
  \querya (Q1 (A R)) \\
)
}
\end{frame}

%---------------------------------------------------------------------

\begin{frame}{Example --- Multiple queries}

{\footnotesize
\expr{%
(\checksys NonDetArbiter
  \inputa ((r1 Bool) (r2 Bool)) \\[.8ex]
~~\outputa ((g1 Bool) (g2 Bool)) \\[.8ex]
~~\assumptiona (a (not (and r1 r2))) \\[.8ex]
~~\comm{; Neg of:~Every request is (immediately) granted} \\
~~\reachablea (p1 (not (and (=> r1 g1) (=> r2 g2)))) \\[.8ex]
~~\comm{; Neg of:~In the absence of other requests, every request is granted} \\
~~\reachablea (p2
      (not (=> (!= r1 r2) 
          (and (=> r1 g1) (=> r2 g2))))) \\[.8ex]
~~\comm{; Neg of:~A request is granted only if present} \\
~~\reachablea (p3 
      (not (and (=> g1 r1) (=> g2 r2)))) \\[.8ex]
~~\comm{; Neg of:~At most one request is granted at any one time} \\
~~\reachablea (p4 
      (not (not (and g1 g2)))) \\[.8ex]
~~\comm{; Neg of:~In case of concurrent requests, one of them is always granted} \\
~~\reachablea (p5 
      (not (=> (and r1 r2) (or g1 g2)))) \\[.8ex]
~~\querya (q1 (a p1))
  \querya (q2 (a p2))
  \querya (q3 (a p3)) \\
~~\querya (q4 (a p4))
  \querya (q5 (a p5)) \\
)
}
}
{\small
\medskip

Each query can be witnessed 
by a \alert{different} $\T$-interpretation and trace in it
}

\end{frame}

%---------------------------------------------------------------------

\begin{frame}{\alert{Output format} for \checksys}
\smallskip
\footnotesize

\expr{%
(\definesys $A$
  \inputa  (($i$ $\sigma_A$))
  \outputa (($o$ $\tau_A$)) 
  \locala (($s$ $\theta_A$))
  $\ldots$
)
\ \\[1.3ex]
(\definesys $B$
  \inputa  (($i$ $\sigma_B$))
  \outputa (($o$ $\tau_B$))
  \locala (($s$ $\theta_B$)) \\
~\subsysa ( $\cdots$ (S (A $\cdots$)) $\cdots$) $\ldots$
)
\ \\[1.2ex]
(\checksys $B$ ~$\cdots$~
 \fairnessa ($f$ $\cdots$)~
 \reachablea ($r$ $\cdots$)~
 $\cdots$~ \\
~\querya ($q$ ($r$ $f$ $\cdots$)) $\cdots$  )\\ %\querya ($q_n$ ($r_{n,1}$ $\cdots$)))\\
}
\bigskip
Output:
\medskip

\expr{%
(\checkres \\
~\querya (q :result sat :model m :trace t) \\
~\modela ($\:\cdots$) ~~~~~~~~\;\comm{; SMT-LIB interpretation of free symbols} \\
~\traila ($p$ (\comm{; state sequence} \\
~~~~~~~~~~~~(($i$ $i_0$) ($o$ $o_0$) ($s$ $s_0$)
                  ($S$::$i$ $i_{S,0}$) ($S$::$o$ $o_{S,0}$) ($S$::$s$ $s_{S,0}$) ($r$ $r_0$) ($f$ $f_0$)$\cdots$) \\
~~~~~~~~~~~~~~~~$\cdots$ \\
~~~~~~~~~~~~(($i$ $i_k$) ($o$ $o_k$) ($s$ $s_k$)
                  ($S$::$i$ $i_{S,k}$) ($S$::$o$ $o_{S,k}$) ($S$::$s$ $s_{S,k}$) ($r$ $r_k$) ($f$ $f_k$) $\cdots$) \\
~~~~~~~~~~~)\\
~~~~~~~~) \\
~\traila ($l$ ( $\cdots$ )) \\
%~:\traila ($l$ ((($i$ $i_{n}$) ($o$ $o_{n}$) ($s$ $s_{n}$)
%                  ($S$::$i$ $i_{S,n}$) ($S$::$o$ $o_{S,n}$) ($S$::$s$ $s_{S,n}$) ($r$ $r_n$) $\cdots$)\\
%~~~~~~~~~~~~~~~~$\cdots$ \\
%~~~~~~~~) \\
~~~~~~$\cdots$ \\
~\tracea ($q$ \prefixa $p$ \lassoa $l$) \comm{; witness trace for query $q$ is $pl^\omega$} \\
~$\cdots$ \\
)
}

\end{frame}

%---------------------------------------------------------------------

\begin{frame}{Additional features}
\large

\begin{itemize}\setlength{\itemsep}{3ex}

\item Special predicate for \alert{frame conditions}
 
\item Special predicate for \alert{deadlock states}
 
\item \alert{Aggregate} queries 
\end{itemize}

\end{frame}

%---------------------------------------------------------------------

\begin{frame}{Planned extensions}
\large

\begin{itemize}\setlength{\itemsep}{3ex}

\item \alert{Executable} system definitions 
 
\item \alert{Parametric} models
 
 \end{itemize}

\end{frame}

%---------------------------------------------------------------------

\begin{frame}{Resources}

Available at \url{https://github.com/ModelChecker/FMCAD23-Tutorial}
\smallskip

\begin{itemize}
\item These slides
\item Examples of systems and queries
\item Syntax highlighting for VS Code 
\item Executables of an experimental version of Kind 2 model checker with MCIL front-end
\end{itemize}
\vspace{3ex}

Available at \url{https://github.com/ModelChecker/IL}
\smallskip

\begin{itemize}
\item Detailed document of IL definition
\end{itemize}

\end{frame}

%---------------------------------------------------------------------

\begin{frame}{What's intentionally missing (and why)}
\smallskip

\begin{itemize}
\item
Restrictions to just bit vector types \\[.5ex]
\uncover<2,8>{\empha{Other types are useful!}}
\smallskip

\item
Stronger syntactic restrictions for \inita and \transa formulas \\[.5ex]
\uncover<3,8>{\empha{Should be enforced in the user-facing language}}
\smallskip

\item
Direct support for LTL, or your favorite temporal logic, in \checksys \\[.5ex]
\uncover<4,8>{\empha{Generality, mostly}}
\smallskip

\item
Global (mutable) variables \emph{a la} SAL \\[.5ex]
\uncover<5,8>{\empha{Tricky to get right}}
\smallskip

\item
Parametric components as in SMV or SAL \\[.5ex] 
\uncover<6,8>{\empha{Some support. The rest is better provided in the user-facing language }}
\smallskip

\item
Compositional reasoning features
(i.e., assume-guarantee contracts) \\[.5ex]
\uncover<7,8>{\empha{Too many different approaches out there}}
\smallskip

%\item
%Commands for getting the results of the analysis \\[.5ex]
%\uncover<8->{\empha{Currently working on them!}}
\end{itemize}

\end{frame}

%---------------------------------------------------------------------

\begin{frame}{}
\LARGE
\centering
\bigskip

\textbf{Additional Features}

\end{frame}

%---------------------------------------------------------------------

\begin{frame}{Special predicate:\, \expr{Deadlock}}

For every system 
$S = (\,
  I_S[\vec i, \vec o, \vec s],\: 
  T_S[\vec i, \vec o, \vec s, \vec{i'}, \vec{o'}, \vec{s'}]
 \,)
$
\medskip

\expr{Deadlock} is a predicate (implicitly) over $\vec i, \vec o, \vec s$
\bigskip
\medskip
\pause

\begin{center}
A state
$\{
 \vec i \mapsto \vec i_0,\, 
 \vec o \mapsto \vec o_0,\,
 \vec s \mapsto \vec s_0 
\}$ 
satisfies \expr{Deadlock}, or

\df{is deadlocked},
\smallskip

iff
\smallskip

it satisfies the formula \ 
$\exists \vec{i'}\, \forall \vec{o'}\, \forall \vec{s'}\, 
 \lnot T_S[\vec i,\vec o,\vec s,\vec{i'},\vec{o'},\vec{s'}]
$
\end{center}
\end{frame}

%---------------------------------------------------------------------

\begin{frame}{Uses of \ \expr{Deadlock}}

\textbf{\large Examples}
\bigskip

\begin{itemize}
\item
\expr{%
(\checksys $S$ $\,\cdots$ \\
\quad \assumptiona ($a$ $A$) \;\currenta ($d$ Deadlock) \;\querya ($a$ $d$))
} \\[.7ex]
checks the \alert{existence} of deadlocked states under assumption $A$
\bigskip
\pause

\item
\expr{%
(\checksys $S$ $\,\cdots$ \\
\quad \assumptiona ($a$ $A$) \;\reachablea ($d$ Deadlock) \;\querya ($a$ $d$))
} \\[.7ex]
checks the \alert{reachability} of deadlocked states under assumption $A$
\bigskip
\pause

\item
\expr{%
(\checksys $S$ $\,\cdots$ \\
\quad \fairnessa ($f$ true) \;\reachablea ($r$ $R$) \;\querya ($f$ $r$))
} \\[.7ex]
checks the \alert{reachability} of $R$ on \alert{infinite} 
(hence deadlock-free) traces
\end{itemize}
\end{frame}

%---------------------------------------------------------------------

\begin{frame}{Special predicate:\, \expr{OnlyChange}}

For every system 
$S = (\,
  I_S[\vec i, \vec o, \vec s],\: 
  T_S[\vec i, \vec o, \vec s, \vec{i'}, \vec{o'}, \vec{s'}]
 \,)
$
\bigskip

\expr{OnlyChange} is a \emph{multi-arity} predicate over $\vec o, \vec s, \vec{o'}, \vec{s'}$:

\begin{eqnarray*}
 \expr{OnlyChange}(x_1, \ldots, x_n) & \equiv &
 \bigwedge 
 \{ y' = y \mid 
    y \in (\vec o \cup \vec s \cup \vec{o'} \cup \vec{s'}) \setminus 
    \{ x_1, \ldots, x_n \}
 \}
\end{eqnarray*}


\alert{Fixes} the value of all \alert{output and local} variables 
\alert{not in} $(x_1, \ldots, x_n)$
\bigskip
\bigskip
\pause

It is a useful \alert{abbreviation} in transition conditions to express 
transitions \\
that leave many state variables unchanged 

\bigskip
\pause

\textbf{Note:}
$\expr{OnlyChange}(x_1, \ldots, x_n)$ does not actually constrain the $x_i$'s
in any way

\end{frame}

%---------------------------------------------------------------------

\begin{frame}{Example}
\small

\expr{%
\comm{; increment $n_i$ iff $n = i$; $n_i$ is 0 initially if not incremented} \\[.7ex]
(\definesys Increment 
  \inputa ((i Int)) \\
~~\outputa ((inc Bool) (n1 Int) (n2 Int) $\,\cdots$ (n5 Int)) \\
~~\inva (= inc (<= 1 i 5)) \\
~~\inita (and \\
~~~~(=> (= n 1) (and (= n1 1) (= n2 n3 n4 n5 0))) \\
~~~~~$\vdots$ \\
~~~~(=> (= n 5) (and (= n5 1) (= n1 n2 n3 n4 0))) \\
~~~~(=> (not (<= 1 n 5)) (= n1 n2 n3 n4 n5 0)) \\
~~) \\
~~\transa (and \\
~~~~(=> (= n' 1) (and (= n1' (+ n1 1)) (OnlyChange inc n1))) \\
~~~~~$\vdots$ \\
~~~~(=> (= n' 5) (and (= n5' (+ n5 1)) (OnlyChange inc n5))) \\
~~~~(=> (not (<= 1 n' 5)) (OnlyChange inc)) \\
~~) \\
)
}

\end{frame}

%---------------------------------------------------------------------

\begin{frame}[t]{\alert{Aggregate} queries}

\expr{%
(\checksys $S$ \\
\begin{tabular}{ll}
 \inputa 
   & (\,($i_1$ $\mi{\delta}_1$) $\,\cdots$ ($i_m$ $\mi{\delta}_m$)\,) \\
 \outputa
   & (\,($o_1$ $\mi{\tau}_1$) $\,\cdots$ ($o_n$ $\mi{\tau}_n$)\,) \\
 $\vdots$ \\  
   \alert{\queriesa} & (($q_1$ ($g_{1,1}$ $\,\cdots$ $g_{1,n_1}$)) $\cdots$ ($q_k$ ($g_{k,1}$ $\,\cdots$ $g_{k,n_k}$)))
% }%
\end{tabular} \\
)
}
\bigskip


\begin{itemize}
\item Each query $q_i$ can be witnessed by a \alert{different trace}
\item However, each free immutable symbol has the \alert{same interpretation} \\
      across all queries
\end{itemize}

\end{frame}

%---------------------------------------------------------------------

\begin{frame}{}
\LARGE
\centering
\bigskip

\textbf{Possible Extensions}

\end{frame}

%---------------------------------------------------------------------

\begin{frame}{\alert{Executable} system definitions}

%\textbf{Example}
%\medskip

Local and output variables are defined exclusively equationally
\medskip

{\footnotesize

\expr{%
(\definesys TimedSwitch 
  \inputa ((press Bool)) \outputa ((sig Bool)) \\
~~\locala ((s LightStatus) (n Int)) \\
~~\attr{inv-def} ( \\
~~~~(sig (= s On)) \\
~~) \\
~~\attr{init-def} ( \\
~~~~(n 0) \\
~~~~(s (ite press On Off)) \\
~~) \\
~~\attr{next-def} ( \\
~~~~(s' (ite press' (ite (= s Off) On Off)) \\
~~~~~~~~~~(ite (= s Off) Off (ite (< n 10) On Off)))) \\
~~~~(n' (ite (or (= s Off) (s' Off)) 0 (+ n 1))) \\
~~))
}
}
\medskip

\small
\textbf{Restrictions:} (guaranteeing progressiveness and executability)
\begin{itemize}
\item Each local or output variable must be listed in \expr{\attr{inv-def}} or \\
      in both \expr{\attr{init-def}} and \expr{\attr{next-def}}
\item No definitional cycles
\item No uninterpreted symbols
\end{itemize}

\end{frame}

%---------------------------------------------------------------------

\begin{frame}{\alert{Parametric} definitions}
%\textbf{Example}
%\medskip
%
\footnotesize

\expr{%
(\definesys Delay
  \alert{\parama} ((\alert{V} Sort) (\alert{d} \alert{V}) (\alert{n} Int))
  \inputa ((in \alert{V})) \\
~~\outputa ((out \alert{V})) \\
~~\locala ((a (Array Int \alert{V}))) \\
~~\inva (and \\
~~~~(= in (select a 0)) \\
~~~~(= out (select a \alert{n})) \\
~~) \\
~~\inita (forall ((i Int)) (=> (<= 1 i \alert{n}) \\
~~~~~~~~~~~(= (select a i) \alert{d})) \\
~~~~~~~~) \\
~~\transa (forall ((i Int)) (=> (<= 1 i \alert{n})) \\
~~~~~~~~~~~~(= (select a' i) (select a (- i 1)))) \\
~~~~~~~~~) \\
) \\
\ \\
(\checksys Delay
  \alert{\parama} ((\alert{V} String) (\alert{d} "") (\alert{n} 4))
  \inputa ((in String)) \\
~~\outputa ((out String)) \\
~~\locala ((a (Array Int String))) \\
~~\ldots \\
)
}
\medskip

\small
\textbf{Restrictions:}
parameters are immutable (rigid)

\end{frame}

%---------------------------------------------------------------------

\end{document}
